\documentclass[10pt,a4paper]{book}

\usepackage[utf8]{inputenc}
\usepackage[french]{babel}
\usepackage[T1]{fontenc}
\usepackage{amsmath}
\usepackage{amsfonts}
\usepackage{amssymb}
\usepackage{hyperref}
\usepackage{graphicx}
\usepackage{caption}
\usepackage{subcaption}
\usepackage{dirtree}
\usepackage[left=2cm,right=2cm,top=2cm,bottom=2cm]{geometry}

\title{Journal de bord}
\author{Gabin Serrurot}

\begin{document}

\maketitle

\tableofcontents

\part{Journal de bord}

\chapter{Janvier}

\begin{enumerate}
    \item \textbf{08/01/2024:}
        \begin{itemize}
            \item J'ai lu le dossier de présentation du projet
            \item J'ai essayé de créer une organisation sur Github afin d'avoir tous les codes au même endroit et afin d'éviter d'être dépend les uns des autres
        \end{itemize}
    \item \textbf{09/01/2024:}
        \begin{itemize}
            \item J'ai finalement décidé de ne pas faire une organisation, juste de faire un dossier dans mon compte Github et accorder l'accès à Yahya et Andréa afin de tout simplifier. De cette manière, je peux utiliser Github Desktop
            \item J'ai commencé le diagramme des cas d'utilisation "Aller à la recherche d’un actif dans l’entrepôt" avec le site \href{https://online.visual-paradigm.com}{visual-paradigm} : figure~\ref{recherche-actif.png1}
            \item J'ai commencé le diagramme des cas d'utilisation "Contrôler le chargement du camion" avec le site \href{https://online.visual-paradigm.com}{visual-paradigm} : figure~\ref{controle-chargement.png1}
            \item Il y aura aussi les diagrammes d'exigences et de déploiement à faire, même s'ils ne sont pas notés dans le dossier de présentation du projet
        \end{itemize}
    \item \textbf{10/01/2024:}
        \begin{itemize}
            \item J'ai d'abord perdu une bonne heure à essayer de debugger mon journal de bord, l'inclusion d'images ne fonctionnait pas alors qu'il manquait juste "usepackage{graphicx}" pour que tout fonctionne.
            \item J'ai repris les diagrammes que j'ai fais la veille avec M. Hacquard : figures~\ref{recherche_actif.png2} et~\ref{controle_chargement.png2} 
        \end{itemize}
    \item \textbf{11/01/2024:}
        \begin{itemize}
            \item Je me suis lancé dans les diagrammes de déploiement mais Pierre m'a dit que ces diagrammes sont à faire par équipe car ils représentent tout le système, pas chacun le sien
            \item J'ai donc décidé de faire la reformulation du cahier des charges
        \end{itemize}
    \item \textbf{12/01/2024:}
        \begin{itemize}
            \item J'ai vu pour la première fois un beacon avec M. Lejoncour
            \item J'ai commencé le cahier de recettes
        \end{itemize}
    \item \textbf{15/01/2024:}
        \begin{itemize}
            \item J'ai fais l'IHM de l'application (la veille mais faut pas le dire, cf image~\ref{idee_design_application1})
            \item J'ai continué le cahier de recettes
            \item J'ai aidé Andréa à installer \textbf{JMerise}
            \item "Définir le plan de numérotation des tags" devient "Donner un tableau exemple d'association entre les beacons et les actifs" dans \textbf{Planification des tâches du projet}
        \end{itemize}
    \item \textbf{16/01/2024:}
        \begin{itemize}
            \item Avec Yahya et Andréa nous avons fait le diagramme de déploiement (cf image~\ref{diagramme_deploiement}) et le diagramme d'exigences  (cf image~\ref{diagramme_exigences1})
            \item Le diagramme de déploiement convient bien mais par le diagramme d'exigences
            \item Pour l'IHM, je ne ferai probablement pas de système de connexion car ça rajouterait des contraintes trop importantes en terme de charge de travail, je me contenterai probablement plus d'un bouton de type "switch" pour basculer entre le mode de contrôle du chargement et le mode de recherche d'actif dans l'entrepôt
        \end{itemize}
    \item \textbf{17/01/2024:}
        \begin{itemize}
            \item Nous allons reprendre le diagramme d'exigences. En parlant avec M. Hacquard, nous avons fait deux diagrammes d'exigences, un pour le camion (cf image~\ref{diagramme_exigences_camion}) et un pour l'entrepôt (cf image~\ref{diagramme_exigences_entrepot})
            \item Nous avons également légèrement modifié le diagramme de déploiement (cf image~\ref{diagramme_deploiement2})
        \end{itemize}
    
    \newpage
    
    \item \textbf{18/01/2024:}
        \begin{itemize}
            \item Je vais commencer mes fiches de test car M. Hacquard a dit que les exigences en périphérie du diagramme d'exigences représentent les tests unitaires et les exigences plus hautes dans la hiérarchie du diagramme représentent les tests d'intégration
            \item J'ai réagencer le diagramme d'exigences de l'entrepôt pour améliorer la lisibilité
            \item J'ai perdu énormément de temps car \textbf{Github Desktop} a fait des siennes, il ne m'affichait plus les commits donc je ne pouvais pas faire de push sur mon Github. Je suis sorti de la salle sans que le problème soit résolu, espérons que ça ira mieux demain
        \end{itemize}
    \item \textbf{19/01/2024:}
        \begin{itemize}
            \item A priori Github Desktop fonctionne de nouveau, j'ai quand même pris du temps pour reprendre là où jen étais hier
            \item On a convenu avec M. Le Joncour que nous feront notre première revue non formelle vendredi prochain, le 26 janvier donc il faudra que nous fassions notre diapo lundi ou mardi prochain afin d'être pret en avance. Il s'agit d'une revue non formelle donc il n'y aura normalement pas de partie individuelle, il y a juste la présentation en groupe à faire. Pour cette revue nous attendons que Andréa et Yahya aient finis le modèle logique de données
            \item J'ai réalisé le cahier plan de numérotation des tags
            \item J'ai poursuivi les fiches de tests
        \end{itemize}
    \item \textbf{20/01/2024:}
        \begin{itemize}
            \item J'ai terminé (sur mon temps libre oui) les fiches de tests
        \end{itemize}
    \item \textbf{22/01/2024:}
        \begin{itemize}
            \item J'étais en CCF, je n'ai rien pu faire du tout je suis sorti 10 minutes avant la fin
        \end{itemize}
    \item \textbf{23/01/2024:}
        \begin{itemize}
            \item Nous avons d'abord décidé de ce que nous allions mettre dans nos diapos
        \end{itemize}
    \item \textbf{24/01/2024:}
        \begin{itemize}
            \item J'ai poursuivi les diapos
            \item Nous avons perdu environ 1h15 car nous avions une présentation sur la poursuite d'études au Québec
        \end{itemize}
    \item \textbf{25/01/2024:}
        \begin{itemize}
            \item J'ai poursuivi les diapos mais pendant peu de temps, j'ai terminé à la maison
            \item Nous avons passé beacoup de temps avec M. Hacquard car nous avons essayé une nouvelle alternative à notre MCD: notre tentative (image~\ref{MLD_proposition_nous}) et la proposition (image~\ref{MLD_proposition_hacquard})
        \end{itemize}
    \item \textbf{26/01/2024:}
        \begin{itemize}
            \item Nous sommes passé pour la première revue, non notée et informelle
        \end{itemize}
    \item \textbf{29/01/2024:}
        \begin{itemize}
            \item Les débuts ont été très compliqués, la connexion a été infernale et mes données mobiles également
            \item J'ai créé le projet sur \textbf{Android-studio} comme étant un \textbf{navigation drawer activity}
            \item J'ai commencé à créer le Figma
        \end{itemize}
    \item \textbf{30/01/2024:}
        \begin{itemize}
            \item J'ai fini \href{https://www.figma.com/proto/0Jh7ikCgIu2jUc2U5tIpNI/Tracking-BLE?type=design&node-id=1-2&t=LAlJHjfGajoQFf8A-0&scaling=scale-down&page-id=0%3A1&starting-point-node-id=1%3A2}{Figma} : image~\ref{figma_v1}
            \item J'ai essayé de lancer le projet sur Android studio avant de réellement commencer à coder mais je n'ai pas réussi, la première compilation a pris plus de 20 minutes sans aboutir
        \end{itemize}
    \item \textbf{31/01/2024:}
        \begin{itemize}
            \item J'ai essayé d'installer \textbf{JetBrains Toolbox} afin d'installer \textbf{Android Studio Giraffe} et non \textbf{Android studio Hérisson}, l'objectif étant de voir si la version trop récente aurait pu expliquer les problèmes de création de projet
            \item A priori la solution ne venait pas uniquement de là, j'ai créé un dossier dans mes documents et fais un lien avec github
            \item J'en ai également profité pour refaire mon architecture dossier du projet. J'ai tout mis dans mes documents avec cette architecture:
            \begin{center}
                \begin{verbatim}
                    Documents/
                        Tracking-BLE/
                            Tracking-BLE-Code/
                                Le reste
                            Tracking-BLE-documents/
                                Le reste
                \end{verbatim}
            \end{center}
        \end{itemize}
\end{enumerate}

\chapter{Février}

\begin{enumerate}
    \item \textbf{01/02/2024:}
        \begin{itemize}
            \item J'ai enfin commencé le développement
            \item J'ai essayé de mettre l'application provenant de \href{https://github.com/Gabin221/android-beacon-library-reference-kotlin}{github} sur mon téléphone afin d'avoir une balise fonctionnelle en permanance
            \item J'ai rempli la base de données avec des valeurs random pour soulager Yahya
        \end{itemize}
    \item \textbf{02/02/2024:}
        \begin{itemize}
            \item J'ai modifié les fichiers du projet de l'application afin d'avoir les noms de fichier corrects, les bons logos dans le menu et les bonnes redirections entre chaque page. Désormais, l'application ressemble visuellement à ce que je voulais
        \end{itemize}
    \item \textbf{05/02/2024:}
        \begin{itemize}
            \item J'ai ajouté un séparateur entre les éléments du menu afin d'isoler les paramètres
            \item J'ai ajouté une table \textbf{parametres} afin d'ajouter en base de données certains paramètres de l'application: 
                \begin{itemize}
                    \item major pour trier quels beacons essayer de détecter
                \end{itemize}
            \item J'ai modifié l'image du logo de l'application pour avoir le logo de l'entreprise
            \item J'ai essayé de commencer à remplir la page des paramètres mais le problème est la difficulté apparente à créer des champs de saisie ainsi que du texte à afficher
        \end{itemize}
    \item \textbf{06/02/2024:}
        \begin{itemize}
            \item J'ai essayé de commencer à remplir la page des paramètres mais le problème est la difficulté apparente à créer des champs de saisie ainsi que du texte à afficher
            \item J'ai donc essayé d'utiliser le code de \href{https://github.com/Gabin221/android-beacon-library-reference-kotlin}{l'application} afin de commencer à remplir mon application mais je me suis rendu compte que l'utilisation de ces codes allait être vraiment compliquée, par exemple le code semble appeler une classe ou un fichier \textbf{beacon} qui semble ne pas exister. M. Le Joncour ne semble pas plus comprendre que moi et nous avons galéré à comprendre ce qui se passe, en vain
        \end{itemize}
    \item \textbf{07/02/2024:}
        \begin{itemize}
            \item Je vais essayer de nouveau de comprendre comment incorporer le code de l'appli dans le miens en espérant avoir un meilleur succès que hier
        \end{itemize}
    \item \textbf{08/02/2024:}
        \begin{itemize}
            \item Je n'ai toujours pas plus réussi hier donc je vais essayer de retrouver moi-même comment détecter les beacons environnants afin de l'intégrer au mieux dans mon application
        \end{itemize}
    \item \textbf{09/02/2024:}
        \begin{itemize}
            \item Finalement M. Le Joncour a réussi à me montrer comment utiliser la recherche de beacons grâce à \href{https://altbeacon.github.io/android-beacon-library/configure.html}{ce lien} donc je vais normalement pouvoir garder mon application de base
            \item Le seul problème va être de découvrir comment écrire dedans, comment passer de 2 fichiers par page à un seul
        \end{itemize}
    \item \textbf{12/02/2024:}
        \begin{itemize}
            \item Grâce à mon application perso, je sais maintenant passer de deux fichiers à un seul donc je vais essayer finalement de remplir en premier lieu la page du camion
            \item J'ai obtenu l'application de M. Le Joncour permettant de détecter les beacons environnants: \textbf{app\_basic\_view}
        \end{itemize}
\end{enumerate}

\chapter{Mars}

\begin{enumerate}
    \item 
\end{enumerate}

\chapter{Avril}

\begin{enumerate}
    \item 
\end{enumerate}

\chapter{Mai}

\begin{enumerate}
    \item 
\end{enumerate}

\chapter{Juin}

\begin{enumerate}
    \item 
\end{enumerate}

\newpage

\part{Annexes}

\begin{figure}[h!]
    \centering
    \begin{subfigure}[b]{0.45\textwidth}
        \centering
        \includegraphics[scale=0.14]{Images/recherche-actif.png}
        \caption{}
        \label{recherche-actif.png1}
    \end{subfigure}
    \begin{subfigure}[b]{0.45\textwidth}
        \includegraphics[scale=0.14]{Images/controle-chargement.png}
        \caption{}
        \label{controle-chargement.png1}
    \end{subfigure}
    \caption{}
\end{figure}

\begin{figure}[h!]
    \centering
    \begin{subfigure}[b]{0.45\textwidth}
        \centering
        \includegraphics[scale=0.14]{Images/recherche_actif.png}
        \caption{}
        \label{recherche_actif.png2}
    \end{subfigure}
    \begin{subfigure}[b]{0.45\textwidth}
        \includegraphics[scale=0.14]{Images/controle_chargement.png}
        \caption{}
        \label{controle_chargement.png2}
    \end{subfigure}
    \caption{}
\end{figure}

\begin{figure}[h!]
    \centering
    \includegraphics[scale=0.2]{Images/idee_design_application1.png}
    \caption{}
    \label{idee_design_application1}
\end{figure}

\begin{figure}[h!]
    \centering
    \begin{subfigure}[b]{0.45\textwidth}
        \centering
        \includegraphics[scale=0.14]{Images/diagramme_deploiement.png}
        \caption{}
        \label{diagramme_deploiement}
    \end{subfigure}
    \begin{subfigure}[b]{0.45\textwidth}
        \includegraphics[scale=0.14]{Images/diagramme_exigence1.png}
        \caption{}
        \label{diagramme_exigences1}
    \end{subfigure}
    \caption{}
\end{figure}

\begin{figure}[h!]
    \centering
    \begin{subfigure}[b]{0.45\textwidth}
        \centering
        \includegraphics[scale=0.14]{Images/diagramme_exigences_camion.png}
        \caption{}
        \label{diagramme_exigences_camion} 
    \end{subfigure}
    \begin{subfigure}[b]{0.45\textwidth}
        \includegraphics[scale=0.14]{Images/diagramme_exigences_entrepot.png}
        \caption{}
        \label{diagramme_exigences_entrepot}
    \end{subfigure}
    \caption{}
\end{figure}

\begin{figure}[h!]
    \centering
    \includegraphics[scale=0.2]{Images/diagramme_deploiement2.png}
    \caption{}
    \label{diagramme_deploiement2}
\end{figure}

\begin{figure}[h!]
    \centering
    \begin{subfigure}[b]{0.45\textwidth}
        \centering
        \includegraphics[scale=0.14]{Images/MLD_proposition_nous.png}
        \caption{}
        \label{MLD_proposition_nous} 
    \end{subfigure}
    \begin{subfigure}[b]{0.45\textwidth}
        \includegraphics[scale=0.14]{Images/MLD_proposition_hacquard.jpg}
        \caption{}
        \label{MLD_proposition_hacquard}
    \end{subfigure}
    \caption{}
\end{figure}

\begin{figure}[h!]
    \centering
    \includegraphics[scale=0.2]{Images/figma_v1.png}
    \caption{}
    \label{figma_v1}
\end{figure}

\end{document}
