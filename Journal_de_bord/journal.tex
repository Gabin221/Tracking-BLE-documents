\documentclass[10pt,a4paper]{article}

\usepackage[utf8]{inputenc}
\usepackage[french]{babel}
\usepackage[T1]{fontenc}
\usepackage{amsmath}
\usepackage{amsfonts}
\usepackage{amssymb}
\usepackage{hyperref}
\usepackage[left=2cm,right=2cm,top=2cm,bottom=2cm]{geometry}

\title{Journal de bord}
\author{Gabin Serrurot}

\begin{document}

\maketitle

\begin{enumerate}
    \item \textbf{08/01/2024:}
        \begin{itemize}
            \item J'ai lu le dossier de présentation du projet
            \item J'ai essayé de créer une organisation sur GitHub afin d'avoir tous les codes au même endroit et afin d'éviter d'être dépend les uns des autres
        \end{itemize}
    \item \textbf{09/01/2024:}
        \begin{itemize}
            \item J'ai finalement décidé de ne pas faire une organisation, juste de faire un dossier dans mon compte GitHub et accorder l'accès à Yahya et Andréa afin de tout simplifier. De cette manière, je peux utiliser GitHub Desktop
            \item J'ai commencé le diagramme des cas d'utilisation "Aller à la recherche d’un actif dans l’entrepôt" avec le site \href{https://online.visual-paradigm.com/drive/#diagramlist:proj=0&dashboard}{visual-paradigm}
            \item J'ai commencé le diagramme des cas d'utilisation "Contrôler le chargement du camion" avec le site \href{https://online.visual-paradigm.com/drive/#diagramlist:proj=0&dashboard}{visual-paradigm}
            \item Diagrammes d'exigences et de déploiement
        \end{itemize}
\end{enumerate}

\end{document}
